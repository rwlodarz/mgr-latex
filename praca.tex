\documentclass[11pt]{aghdpl}
% \documentclass[en,11pt]{aghdpl}  % praca w języku angielskim
\usepackage[polish]{babel}
%\usepackage[english]{babel}
\usepackage[utf8]{inputenc}

% dodatkowe pakiety
\usepackage{enumerate}
\usepackage{listings}
\lstloadlanguages{TeX}

\lstset{
  literate={ą}{{\k{a}}}1
           {ć}{{\'c}}1
           {ę}{{\k{e}}}1
           {ó}{{\'o}}1
           {ń}{{\'n}}1
           {ł}{{\l{}}}1
           {ś}{{\'s}}1
           {ź}{{\'z}}1
           {ż}{{\.z}}1
           {Ą}{{\k{A}}}1
           {Ć}{{\'C}}1
           {Ę}{{\k{E}}}1
           {Ó}{{\'O}}1
           {Ń}{{\'N}}1
           {Ł}{{\L{}}}1
           {Ś}{{\'S}}1
           {Ź}{{\'Z}}1
           {Ż}{{\.Z}}1
}

%---------------------------------------------------------------------------

\author{Rafał Włodarz}
\shortauthor{R. Włodarz}

\titlePL{Algorytm sterowania wykorzystujący sztuczne sieci neuronowe dla bezzałogowego statku latającego typu TRICOPTER}
\titleEN{}

\shorttitlePL{Algorytm sterowania wykorzystujący sztuczne sieci neuronowe} % skrócona wersja tytułu jeśli jest bardzo długi
%\shorttitleEN{Thesis in \LaTeX}

\thesistype{Praca dyplomowa magisterska}
%\thesistype{Master of Science Thesis}

\supervisor{dr hab. Adam Piłat}
%\supervisor{Marcin Szpyrka PhD, DSc}

\degreeprogramme{Automatyka i robotyka}
%\degreeprogramme{Computer Science}

\date{2015}

\department{Katedra Automatyki}
%\department{Department of Applied Computer Science}

\faculty{Wydział Elektrotechniki, Automatyki,\protect\\[-1mm] Informatyki i Inżynierii Biomedycznej}
%\faculty{Faculty of Electrical Engineering, Automatics, Computer Science and Biomedical Engineering}

\acknowledgements{Serdecznie dziękuję \dots tu ciąg dalszych podziękowań np. dla promotora, żony, sąsiada itp.}


\setlength{\cftsecnumwidth}{10mm}

%---------------------------------------------------------------------------
\setcounter{secnumdepth}{4}

\begin{document}

\titlepages
\setcounter{tocdepth}{3}
\tableofcontents
\clearpage

\chapter{Wstęp}
\label{cha:wprowadzenie}

\section{Cele pracy}
\label{sec:celePracy}

\section{Zawartość pracy}
\label{sec: zawartosc_pracy}



















\chapter{Sztuczne sieci neuronowe}
\label{cha:sztuczne_sieci_neuronowe}

Rozdział ten zawiera informacje na temat sieci neuronowych, ich architektury, zasady działania oraz algorytmów uczenia.

\section{Początki sieci neuronowych}
\label{sec:poczatki_sieci_neuronowych}
Początki prac nad poznaniem procesów zachodzących w mózgu datuje się na rok 1943. W pracy McCulloch'a oraz Pitts'a przedstawiono matematyczny model neuronu, który zapoczątkował badania związane z tym tematem. W 1949 roku Donald Hebb odkrył, iż informacje przechowywane w sieci neuronowej są reprezentowane jako wartości wag pomiędzy poszczególnymi neuronami. Na podstawie tych informacji zaproponował on pierwszy algorytm uczenia sieci neuronowej, który został nazwany regułą Hebba. Już wtedy odkryto, iż bardzo dużą zaletą sieci jest równoległy sposób przetwarzania informacji oraz metodologia uczenia, która zastępuje tradycyjny proces programowania. 






% itd.
% \appendix
% \include{dodatekA}
% \include{dodatekB}
% itd.

\bibliographystyle{alpha}
\bibliography{bibliografia}
%\begin{thebibliography}{1}
%
%\bibitem{Dil00}
%A.~Diller.
%\newblock {\em LaTeX wiersz po wierszu}.
%\newblock Wydawnictwo Helion, Gliwice, 2000.
%
%\bibitem{Lam92}
%L.~Lamport.
%\newblock {\em LaTeX system przygotowywania dokumentów}.
%\newblock Wydawnictwo Ariel, Krakow, 1992.
%
%\bibitem{Alvis2011}
%M.~Szpyrka.
%\newblock {\em {On Line Alvis Manual}}.
%\newblock AGH University of Science and Technology, 2011.cccccc
%\newblock \\\texttt{http://fm.ia.agh.edu.pl/alvis:manual}.
%
%\end{thebibliography}

\end{document}
