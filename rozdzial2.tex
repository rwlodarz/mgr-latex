\chapter{Sztuczne sieci neuronowe}
\label{cha:sztuczne_sieci_neuronowe}

Rozdział ten zawiera informacje na temat sieci neuronowych, ich architektury, zasady działania oraz algorytmów uczenia.

\section{Początki sieci neuronowych}
\label{sec:poczatki_sieci_neuronowych}
Początki prac nad poznaniem procesów zachodzących w mózgu datuje się na rok 1943. W pracy McCulloch'a oraz Pitts'a przedstawiono matematyczny model neuronu, który zapoczątkował badania związane z tym tematem. W 1949 roku Donald Hebb odkrył, iż informacje przechowywane w sieci neuronowej są reprezentowane jako wartości wag pomiędzy poszczególnymi neuronami. Na podstawie tych informacji zaproponował on pierwszy algorytm uczenia sieci neuronowej, który został nazwany regułą Hebba. Już wtedy odkryto, iż bardzo dużą zaletą sieci jest równoległy sposób przetwarzania informacji oraz metodologia uczenia, która zastępuje tradycyjny proces programowania. 


