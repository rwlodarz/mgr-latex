\chapter{Aplikacja sterująca}
\label{cha:aplikacja_sterujaca}
W niniejszym rozdziale przedstawiono informacje na temat aplikacji sterującej. Zaprezentowano kolejne etapy przygotowania platformy sprzętowej, które wymagane są do weryfikacji działania systemu czasu rzeczywistego o ostrych ograniczeniach czasowych (ang. hard real-time).

\section{Konfiguracja beaglebone black}

\subsection{System czasu rzeczywistego}
System czasu rzeczywistego (ang. real-time system) to system, który przetwarza każdy rodzaj informacji i który musi reagować na sygnały wejściowe - bodźce generowane z zewnątrz w skończonym i określonym czasie. Jego poprawne działanie zależy zarówno od prawidłowych rezultatów logicznych, jak również od czasu reakcji.
Na podstawie tych kryteriów są one dzielone na:
\begin{itemize}
	\item Systemy o ostrych wymaganiach czasowych (ang. hard real-time) - wymagania czasowe muszą być skrupulatnie przestrzegane, naruszenie ram czasowych może wpłynąć na życie ludzkie, środowisko czy też sam system,
 
	\item Systemy o słabych wymaganiach czasowych (ang. soft real-time) – głównym kryterium oceny tych jest średni czas odpowiedzi. Sporadyczne opóźnienie nie powoduje zagrożenia lecz jedynie wpływa negatywnie na ocenę całego systemu,

	\item Systemy o solidnych wymaganiach czasowych (ang. firm real-time) - są one kombinacją systemów o wymaganiach ostrych oraz słabych. Naruszenie kryterium czasowych może pojawiać się okazjonalnie. Często dla lepszej oceny systemu stosuje się ograniczenia czasowe o charakterze "słabym" - krótsze, których przekroczenie nie powoduje katastrofy oraz "ostrym" - dłuższe, których naruszenie oznacza nieprawidłowe działanie systemu. 

\end{itemize}

Bez względu na to do której grupy systemów zalicza się aplikację musi ona charakteryzować się następującymi cechami:

\begin{itemize}
	\item Ciągłość działania - powinny działać nieprzerwanie w okresie od uruchomienia systemu do jego wycofania,
	
	\item Zależność od otoczenia - zachowanie opiera rozpatruje się w kontekście otoczenia. Prowadzone obliczenia zależą od zdarzeń oraz danych pochodzących z zewnątrz układu,
	
	\item Współbieżność - struktura systemu narzuca, aby jednoczesne zdarzenia były obsługiwane równocześnie przez szereg procesów,
	
	\item Przewidywalność - zdarzenia i dane generowane przez otoczenie pojawiają się przypadkowo co nie narusza deterministycznego zachowania systemu,
	
	\item Punktualność - odpowiedź systemu na bodźce zewnętrzne powinna być dostarczona w odpowiednich momentach - wymaganych ramach czasowych.
\end{itemize}

Z pewnością aplikacja sterująca obiektami latającymi powinna spełniać wszystkie powyższe założenia. Gdyby, któreś z nich nie zostało spełnione jakiekolwiek próby sterowania zakończyły by się porażką.

Dynamika wielokomórkowców wymaga od aplikacji bardzo szybkiego czasu reakcji na zewnętrzne impulsy. Opóźnienie sterowania w takim przypadku powoduje bardzo negatywne skutki do których zaliczamy brak kontroli nad obiektem co zazwyczaj dąży do utraty stabilności w powietrzu, a następnie katastrofy po zetknięciu się z przeszkodą.

Na podstawie powyższych definicji aplikację sterującą zdecydowanie zalicza się do systemu hard-real time. 


\subsection{Przygotowanie systemu operacyjnego}
Ze względu na wcześniej wspominaną specyfikę systemu sterującego oraz zastosowanie platformy sprzętowej typu mini PC wraz z systemem operacyjnym typu UNIX, ważne jest, aby wyeliminować wszelkie możliwe przerwania oraz inne aspekty, które wpływają na płynność oraz czas wykonywania się aplikacji sterującej.

Zostało to zapewnione przez instalację systemu operacyjnego wyposażonego w jądro czasu rzeczywistego (ang. real-time kernel). Jądro to określane jest również jako w pełni wywłaszczalne. Oznacza to, iż dopuszczane jest wywłaszczenie procesu działającego w trybie jądra. Cecha ta wraz z wcześniej narzuconymi priorytetami dla poszczególnych procesów gwarantuje, iż aplikacja sterująca uruchomiona z wysokim priorytetem. Nie zostanie wywłaszczona na zbyt długi czas przez inne procesy systemowe. 

Zabieg ten pozwala nam spełnić podstawową cechę systemów czasu rzeczywistego, którą jest punktualność.


\subsection{Analiza operacji zmiennoprzecinkowych}
Dobrze zaprojektowana aplikacja sterująca statkami latającymi powinna być możliwa do uruchomienia na różnorodnych platformach sprzętowych, bez względu na architekturę wykorzystanych do ich budowy procesorów.

Podstawowa wersja aplikacji została poddana testom, które porównały wyniki działania sieci neuronowej na 2 całkowicie odmiennych platformach. Konfiguracja miała na celu uwydatnienie problemu błędu obcięcia liczb zmiennoprzecinkowych 

\subsubsection{Testy}
Do testów wykorzystano komputer stacjonarny z 64 bitowym procesorem Intel i5-4670 3.40 Ghz oraz układ beaglebone black, który został wyposażony w procesor 32 bitowy ARM Cortex-A8 o częstotliwości taktowania 1GHz.

W celu uwydatnienia problemu stworzono sieć neuronową posiadającą po jednej z warstw wejściowej, ukrytej, wyjściowej. Wagi poszczególnych neuronów zostały dobrane w sposób losowy tak, jednak wszystkich z nich wykorzystują maksymalną dostępną dokładność. Następnie na wejście sieci podawana jest stała wartość. Po jej przetworzeniu otrzymujemy wynik  


\section{Architektura systemu sterującego}
