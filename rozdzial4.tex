\chapter{Aplikacja sterująca}
\label{cha:aplikacja_sterujaca}
W niniejszym rozdziale przedstawiono informacje na temat aplikacji sterującej. Zaprezentowano kolejne etapy przygotowania platformy sprzętowej, które wymagane są do weryfikacji działania systemu czasu rzeczywistego o ostrych ograniczeniach czasowych (ang. hard real-time).

\section{Konfiguracja beaglebone black}

\subsection{System czasu rzeczywistego}
System czasu rzeczywistego (ang. real-time system) to system, który przetwarza każdy rodzaj informacji i który musi reagować na sygnały wejściowe - bodźce generowane z zewnątrz w skończonym i określonym czasie. Jego poprawne działanie zależy zarówno od prawidłowych rezultatów logicznych, jak również od czasu reakcji.
Na podstawie tych kryteriów są one dzielone na:
\begin{itemize}
	\item Systemy o ostrych wymaganiach czasowych (ang. hard real-time) - wymagania czasowe muszą być skrupulatnie przestrzegane, naruszenie ram czasowych może wpłynąć na życie ludzkie, środowisko czy też sam system,
 
	\item Systemy o słabych wymaganiach czasowych (ang. soft real-time) – głównym kryterium oceny tych jest średni czas odpowiedzi. Sporadyczne opóźnienie nie powoduje zagrożenia lecz jedynie wpływa negatywnie na ocenę całego systemu,

	\item Systemy o solidnych wymaganiach czasowych (ang. firm real-time) - są one kombinacją systemów o wymaganiach ostrych oraz słabych. Naruszenie kryterium czasowych może pojawiać się okazjonalnie. Często dla lepszej oceny systemu stosuje się ograniczenia czasowe o charakterze "słabym" - krótsze, których przekroczenie nie powoduje katastrofy oraz "ostrym" - dłuższe, których naruszenie oznacza nieprawidłowe działanie systemu. 

\end{itemize}

Bez względu na to które z powyższych kryteriów są spełniane przez system konieczne jest, aby każdy z nich charakteryzował się następującymi cechami:

\begin{itemize}
	\item Ciągłość działania - powinny działać nieprzerwanie w okresie od uruchomienia systemu do jego wycofania,
	
	\item Zależność od otoczenia - zachowanie opiera rozpatruje się w kontekście otoczenia. Prowadzone obliczenia zależą od zdarzeń oraz danych pochodzących z zewnątrz układu,
	
	\item Współbieżność - struktura systemu narzuca, aby jednoczesne zdarzenia były obsługiwane równocześnie przez szereg procesów,
	
	\item Przewidywalność - zdarzenia i dane generowane przez otoczenie pojawiają się przypadkowo co nie narusza deterministycznego zachowania systemu,
	
	\item Punktualność - odpowiedź systemu na bodźce zewnętrzne powinna być dostarczona w odpowiednich momentach - wymaganych ramach czasowych.
\end{itemize}



\subsection{Przygotowanie systemu operacyjnego}
Ze względu na wcześniej wspominaną specyfikę systemu sterującego oraz zastosowanie platformy sprzętowej typu mini PC wraz z systemem operacyjnym typu UNIX, ważne jest, aby wyeliminować wszelkie możliwe przerwania oraz inne aspekty, które wpływają na płynność oraz czas wykonywania się aplikacji sterującej.




\subsection{Analiza operacji zmiennoprzecinkowych}


\section{Architektura systemu sterującego}
