\chapter{Testy systemu sterującego}
\label{cha:testy}

W tym rozdziale zostały omówione badania sieci neuronowej. Ich głównym celem jest sprawdzenie poprawności jej działania, określenie optymalnej topologii sieci oraz układu połączeń między neuronami.

\section{Testy poprawności działania sztucznej sieci neuronowej}

Ze względu na nietrywialną implementację zagadnień związanych z sieciami neuronowymi oraz algorytmów uczących, sieć została przetestowana pod kątem poprawności działania na zdecydowanie prostszym zagadnieniu jakim jest operacja logiczna typu xor.

Została ona wybrana ze względu na łatwość zrozumienia problemu, bezproblemowe określenie oczekiwanego wyniku oraz brak wymaganego zbioru uczącego - wektor wejściowy może on zostać wygenerowany podczas samego procesu uczenia, natomiast wartość oczekiwana zostaje obliczona przy pomocy prostej operacji logicznej. 